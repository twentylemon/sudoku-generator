% COSC 4P03 Project
% sudoku generator
% Taras Mychaskiw

\section{Future Work}

An obvious extension of the work would be to devise more sudoku generation strategies, in order to produce sudoku puzzles with
fewer clues or produce very hard puzzles. Some additional ideas for generators that were never tested (due to time constraints)
are described below.

\begin{enumerate}
    \item starting from an empty sudoku, select the next clue to place in such a way that it limits the total amount of
    information given to the player, perhaps by maximizing the size of all the candidate sets for each remaining cell

    \item devise some difficulty ratings, and optimize them in the construction of the sudoku puzzle, perhaps a dynamic programming
    algorithm would be possible here to ensure the hardest puzzle in generated
\end{enumerate}

Solving sudoku puzzles does not need additional improvements (in \textit{my} opinion, at least). They are plenty fast, and when run
concurrently, one only reaps the benefits of each solver - the weaknesses of one solver are offset by the strengths of another.
However, if a faster sudoku solver is devised, it can only help the generation of puzzles.

%\end{Future Work}

\clearpage
\section{Conclusions}
All in all, the bottom up generation strategy was the most successful sudoku puzzle generator tested. It will occasionally produce an
effectively useless puzzle, but more often than not it outperforms the other generators in terms of the number of clues in the puzzles
created. It may take slightly more time, but it still created about $20$ sudoku puzzles per second on the computer running the tests.
In the future, one can remove the constraint propagation strategy entirely, and only run backtracking and exact cover solvers while
generating puzzles. This will ensure the unexpected benefits of backtracking and the raw speed of exact cover are both exploited.
%\end{Conclusions}
