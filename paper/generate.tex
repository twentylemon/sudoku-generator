% COSC 4P03 Project
% sudoku generator
% Taras Mychaskiw

\section{Sudoku Generation}

There are several ways of generating a sudoku puzzle. Three approaches are explored here. Each of them produces a random
sudoku puzzle with a unique solution, where if you were to remove any clue, the puzzle would no longer have a unique solution.
Each of the generator algorithms are fairly simple.

\subsection{Top Down Generation}
In top down generation, we start with a randomly solved sudoku board and randomly remove clues from it. With each clue removed,
the puzzle so far is checked to ensure it still has exactly one solution. If it has more than one solution, then the value in the
cell is replaced. This process is done for each cell in the sudoku.
\begin{center}
\begin{pseudocode}[framebox]{topDown}{ }
    sudoku \GETS \text{a solved sudoku board}       \\
    \FOR cell \in sudoku \DO \BEGIN
        \text{remove the cell from sudoku}          \\
        form \GETS \text{formity of sudoku}         \\
        \IF form \neq \text{unique solution} \THEN
            \text{replace the value removed}
    \END                                            \\
    \RETURN {sudoku}
    \label{algo:topdown}
\end{pseudocode}
\end{center}
%\end{Top Down Generation}

\subsection{Bottom Up Generation}
Bottom up generation is the exact opposite of top down generation. Starting with an empty board, randomly add clues
to random cells. Only those clues which are possible to sit in the cell are considered. Once a value is added, the formity
of the sudoku is checked. If there are no solutions, the value added is removed, and the process is continued. If there are
multiple solutions, the process is continued. The process stops once a sudoku with a unique solution is found. However this
produces puzzles with approximately $30$ clues in a $9$x$9$ sudoku, so some post-processing is required. For each clue in the
sudoku, try removing it. If the puzzle still no longer has a unique solution, then replace the value.
\begin{center}
\begin{pseudocode}[framebox]{bottomUp}{ }
    sudoku \GETS \text{an empty sudoku board}                               \\
    \WHILE \TRUE \DO \BEGIN
        cell \GETS \text{a random empty cell} \in sudoku                    \\
        sudoku[cell] \GETS \text{a random possible value}                   \\
        form \GETS \text{formity of sudoku}                                 \\
        \IF form = \text{unique solution} \THEN \EXIT \text{while}          \\
        \IF form = \text{no solution} \THEN \text{clear the cell placed}    \\
    \END    \\
    \FOR clue \in sudoku \DO \BEGIN
        \text{remove the clue from sudoku}          \\
        \IF form \neq \text{unique solution} \THEN
            \text{replace the value removed}
    \END                                            \\
    \RETURN {sudoku}
    \label{algo:bottomup}
\end{pseudocode}
\end{center}
%\end{Bottom Up Generation}

\subsection{Deduction Generation}
Deduction generation was an attempt to produce sudoku puzzles with few clues. Starting with a solved sudoku puzzle, only add
a clue to our returned puzzle if it is not deduced by all the other clues in our return puzzle already. That is, if a cell has
more than $1$ possible value, add the value from the solution to the cell. This also creates a $30$-$40$ clue puzzle, so the same
post-processing as bottom up generation is required.
\begin{center}
\begin{pseudocode}[framebox]{deduction}{ }
    solved \GETS \text{a solved sudoku board}       \\
    sudoku \GETS \text{an empty sudoku board}       \\
    \FOR cell \in sudoku \DO
        \IF |\CALL{getOptions}{sudoku, cell}| > 1 \THEN
            sudoku[cell] \GETS solved[cell]         \\
    \FOR clue \in sudoku \DO \BEGIN
        \text{remove the clue from sudoku}          \\
        \IF form \neq \text{unique solution} \THEN
            \text{replace the value removed}
    \END                                            \\
    \RETURN {sudoku}
    \label{algo:deduction}
\end{pseudocode}
\end{center}
%\end{Deduction Generation}

%\end{Sudoku Generation}
